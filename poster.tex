\RequirePackage[l2tabu, orthodox]{nag}
%\pdfminorversion=4
\documentclass[a0paper,portrait,fontscale=0.35]{baposter}

\usepackage{amsfonts,amsmath,amsthm,amssymb,commath}
\usepackage{graphicx,subcaption}
\usepackage[most,skins,theorems]{tcolorbox}
\usepackage{color}
\usepackage[none]{hyphenat} 

\usepackage{relsize}
\usepackage{natbib}
\usepackage{microtype}

\usepackage{tikz}
\usetikzlibrary{decorations.pathreplacing}

\usepackage{natbib}
\setlength{\bibsep}{0.0pt}

\newcommand{\PSF}{\text{PSF}}
\newcommand{\argmin}{\text{argmin}}

%%% Define a caption font
\newcommand{\mycaption}[1]{
  {
    \smaller
    \emph{#1}
  }
}

\theoremstyle{plain}
\newtheorem{thm}{\protect\theoremname}
  \theoremstyle{plain}
  \newtheorem{lem}[thm]{\protect\lemmaname}
  \theoremstyle{definition}
  \newtheorem{defn}[thm]{\protect\definitionname}
  \theoremstyle{plain}
  \newtheorem{prop}[thm]{\protect\propositionname}
  \theoremstyle{definition}
  \newtheorem{example}[thm]{\protect\examplename}

%%% Color Definitions %%%%%%%%%%%%%%%%%%%%%%%%%%%%%%%%%%%%%%%%%%%%%%%%%%%%%%%%%
%\definecolor{oxford_blue}{RGB}{14,31,71}
%\definecolor{oxford_border}{RGB}{14,31,71}

%\definecolor{cambridge_blue}{RGB}{64,76,0}
%\definecolor{cambridge_blue}{RGB}{163, 193, 173}
%\definecolor{cambridge_border}{RGB}{163, 193, 173}

\definecolor{cambridge_blue}{RGB}{49,72,56}
\definecolor{cambridge_border}{RGB}{49,72,56}

\tcbset{highlight math style={enhanced, 
    colframe=cambridge_blue, colback=white}}

\begin{document}
\typeout{Poster rendering started}

\begin{poster}
{
    % Show grid to help with alignment
    grid=false,
    columns=6,
    % Column spacing
    colspacing=0.7em,
    % Color style
    %headerColorOne=cyan!20!white!90!black,
    %borderColor=cyan!30!white!90!black,
    headerColorOne=cambridge_blue,
    borderColor=cambridge_border,
    headerFontColor=white,
    % Format of textbox
    textborder=faded,
    % Format of text header
    headerborder=open,
    headershape=smallrounded,
    headershade=plain,
    background=none,
    bgColorOne=cyan!10!white,
    %headerheight=0.11\textheight,
    headerheight=0.095\textheight,
    eyecatcher=false,
}
% Eye Catcher: Oxford logo and personal picture
{
%  \makebox[0.23\textwidth]{
%    \begin{tabular}{cc}
%    \includegraphics[height=0.055\textheight]{./img/InFoMM}
%      &
%    %\includegraphics[height=0.10\textheight]{./img/me}
%    \end{tabular}
%    \hfill
%  }
}
%%% Title %%%%%%%%%%%%%%%%%%%%%%%%%%%%%%%%%%%%%%%%%%%%%%%%%%%%%%%%%%%%%%%%%%%%%
{
  %%\textsc{Source Reconstruction \vspace{0.2cm}\\From Hydrophone Data}
  \textsc{Spatially Variable Deconvolution\vspace{0.2cm}\\ 
    for Lightsheet Microscopy\vspace{0.2em}}
  %\vspace{0.3em}
}
%%% Authors %%%%%%%%%%%%%%%%%%%%%%%%%%%%%%%%%%%%%%%%%%%%%%%%%%%%%%%%%%%%%%%%%%%
{
  \vspace{0.1em}
  \hspace{-0.65em}
  {
    \underline{\textbf{Bogdan Toader}}\textsuperscript{1},
    \textbf{J\'{e}r\^{o}me Boulanger}\textsuperscript{2},
    \textbf{Yury Jorolev}\textsuperscript{1},
    \textbf{Martin Lenz}\textsuperscript{1},\\
    \textbf{James Manton}\textsuperscript{2},
    \textbf{Leila Mure\c{s}an}\textsuperscript{1},
    \textbf{Carola-Bibiane Sch\"{o}nlieb}\textsuperscript{1}
  } \\[0.2em]
  {
    \textsuperscript{1}University of Cambridge,
    \textsuperscript{2}MRC Laboratory of Molecular Biology 
  }
  \vspace{-1em}
}
%%% InFoMM Logo %%%%%%%%%%%%%%%%%%%%%%%%%%%%%%%%%%%%%%%%%%%%%%%%%%%%%%%%%%%%%%%
%{
%  %\makebox[0.23\textwidth]{
%  \makebox[0.32\textwidth]{
%    \hfill
%    \begin{tabular}{ccc}
%    %\includegraphics[height=0.055\textheight]{./img/InFoMM}
%      \includegraphics[height=0.04\textheight]{./img/InFoMM}
%      &
%    %\includegraphics[height=0.10\textheight]{./img/me}
%      \includegraphics[height=0.04\textheight]{./img/oxlogo}
%      \\
%      \includegraphics[height=0.04\textheight]{./img/turinglogo2}
%      &
%      \includegraphics[height=0.04\textheight]{./img/edinlogo}
%      %&
%      %\includegraphics[height=0.05\textheight]{./img/me_square}
%    \end{tabular}
%  }
%  %}
%}
{
  %\makebox[0.23\textwidth]{
  \makebox[0.32\textwidth]{
    \hfill
    \begin{tabular}{ccc}
      \includegraphics[height=0.04\textheight]{./img/camlogo}
    \end{tabular}
  }
  %}
}
\headerbox{lightsheet microscopy}{name=lightsheet, span=3, column=0, row=0}{
  Light sheet microscopy is a type of fluorescence 
  microscopy used in cell biology due to its fast 
  acquisition times and low photo-damage to the sample. 
  This is achieved by selectively illuminating a slice 
  of the sample using a sheet of light and detecting the 
  emitted fluorescence signal using a dedicated 
  objective orthogonal to the plane of the sheet. 
  \vspace{1pt}
  \begin{minipage}[t]{0.48\textwidth}
    \centering
    \raisebox{\dimexpr-\height+\ht\strutbox}{
      \includegraphics[width=0.8\textwidth]{img/spim.png}
    }
   
    \mycaption{Figure credit: J\"{o}rg Ritter, PhD thesis (2011)}
  \end{minipage}
  %
  \begin{minipage}[t]{0.48\textwidth}
    \begin{center}
      \larger
      {\color{blue}\textbf{\textsc{Spatially varying PSF}}}\\
      %\textit{$y_i$ are the samples\\
      %  we use to reconstruct $x$.}
    \end{center}
    
    \begin{minipage}[t]{\textwidth}
      \centering
      \includegraphics[width=\textwidth]{img/psf_vary.pdf}
    \end{minipage}
  \end{minipage}
     
  \begin{minipage}[t]{0.48\textwidth}
    \begin{center}
      \larger
      {\color{red}\textbf{\textsc{Problem}}}\\
    \end{center}
    \vspace{-1em}
    \begin{tcolorbox}[colback=red!10!white,colframe=red]
      Due to the interaction of the lightsheet beam  
      and the detection/objective point spread function (PSF), 
      the effective PSF of the system is spatially varying, and
      therefore standard deconvolution approaches are not applicable.
    \end{tcolorbox}
  \end{minipage}
  %
  \hspace{0.6em}
  \begin{minipage}[t]{0.48\textwidth}
    %\vspace{-1em}
    \begin{center}
      \larger
      {\color{blue}\textbf{\textsc{Goal}}}\\
    \end{center}
    \vspace{-1em}
    \begin{tcolorbox}[colback=blue!10!white,colframe=blue]
      In this work, we propose a model for image formation 
      that describes the interaction between the 
      illumination PSF and the detection PSF which 
      replicates the physics of the microscope while 
      leading to a tractable inverse problem.
    \end{tcolorbox}
  \end{minipage}
  \vspace{-1em}
}
\headerbox{psf model}{name=psf, span=3, column=3, row=0}{

  \begin{minipage}[t]{\textwidth}
    The detection PSF $h$ is modelled as the Fourier
    transform of the pupil function multiplied by 
    defocus~\cite{Stokseth1969}: 
    \begin{tcolorbox}[colback=teal!10!white,colframe=white]
      \vspace{-1em}
      \begin{equation}
        h(x,y,z) = \left|
          \iint g_{\sigma} * p(\kappa_x,\kappa_y) e^{
            2 i \pi z 
            \sqrt{(n/\lambda)^2 - \kappa_x^2 - \kappa_y^2}
          }
          e^{
            2 i \pi (\kappa_x x + \kappa_y y)
          }
          \dif \kappa_x \dif \kappa_y
        \right|^2
        \label{eq:psf model}
      \end{equation}
    \end{tcolorbox}
    where $p$ is the pupil function, defined as
    \begin{tcolorbox}[colback=teal!10!white,colframe=white]
      \begin{equation}
        p(\kappa_x,\kappa_y) = \begin{cases}
          e^{2i\pi \sum_{j=1}^{15} c_i Z_j(\kappa_x,\kappa_y)}
          \quad
            &\text{for} \quad
            \rho = \sqrt{\kappa_x^2 + \kappa_y^2} \leq NA/\lambda,\\
            0, \quad &\text{otherwise.}
        \end{cases}
      \end{equation}
    \end{tcolorbox}
    where the phase of the pupil function is computed by least-squares
    fitting of the coefficients of the first $15$ Zernike polynomials
    using an image of a bead.
  \end{minipage}

  \vspace{10pt}
  \begin{minipage}[t]{\textwidth}
    \begin{minipage}[t]{0.38\textwidth}
      The parameters of the model are given by
      the experimental setup used to acquire the image:
      \begin{itemize}
        \item $n$ - refractive index
        \item $\lambda$ - wave length
        \item NA - numerical aperture
        \item $g_{\sigma}$ Gaussian blur to take into account
          other properties not accounted for in our model
      \end{itemize}
    \end{minipage}
    %
    \begin{minipage}[t]{0.3\textwidth}
      \centering
      \raisebox{\dimexpr-\height+\ht\strutbox}{
        \includegraphics[height=0.1\textheight]{img/psf_bead.png}
      }

      \vspace{0.3em}
      \mycaption{
        Bead image (MIP)
      }
    \end{minipage}
    %
    \begin{minipage}[t]{0.3\textwidth}
      \centering
      \raisebox{\dimexpr-\height+\ht\strutbox}{
        \includegraphics[height=0.1\textheight]{img/psf_zernike_sigma.png}
      }

      \vspace{0.3em}
      \mycaption{
        Estimated detection PSF (MIP)
      }
    \end{minipage}
  \end{minipage} 
}

\headerbox{image formation model}{name=model, span=3, column=0, below=lightsheet}{
  %\begin{minipage}[t]{\textwidth} 

    The sample $s$ illuminated at $z=z_0$ by the light-sheet $l$
    and the photons are collected by an objective with PSF $h$:
    \vspace{1em} 
    
    \begin{minipage}[t]{\textwidth}
      \centering
      \raisebox{\dimexpr-\height+\ht\strutbox}{
        \includegraphics[height=0.094\textheight]{img/model_diags.png}
      }
    \end{minipage}

    \begin{tcolorbox}[colback=teal!10!white,colframe=white]
      \vspace{-1em}
      \begin{equation}
        f(x,y,z_0) = \iiint l_{avg_y}(u,v,w) s(u,v,w - z_0) h(x-u,y-v,w) \dif u \dif v \dif w
        \label{eq:lightsheet model}
      \end{equation}
    \end{tcolorbox}
    where $h$ is the detection PSF, calculated using \eqref{eq:psf model} 
    and $l_{avg_y}$ is the light-sheet, calculated by averaging the beam
    PSF obtained in a similar way to $h$, without Zernike polynomials.

    \vspace{0.5em}
    \begin{minipage}[t]{\textwidth}
      \centering
      \raisebox{\dimexpr-\height+\ht\strutbox}{
        \includegraphics[height=0.065\textheight]{img/f_to_s.png}
      }
    \end{minipage}

 % \end{minipage}
}

\headerbox{reconstruction}{name=resultssim, span=3, column=3, below=lightsheet}{
    Let $\hat{f}$ be the image data with Gaussian noise
    and $f(s)$ the result of applying the forward
    model \eqref{eq:lightsheet model} to the sample $s$.
    To recover $s$, we solve:
    \begin{center}
      \tcbhighmath[colback=red!10!white, frame hidden]{
        \text{Find }
        \hat{s} \in 
        \argmin_{s} \left\{
          \| \hat{f} - f(s)   \|_{L_2}
          + \lambda TV(s)  
        \right\}
      }
    \end{center}

    \begin{minipage}[t]{0.55\textwidth}
      \begin{minipage}[t]{0.49\textwidth}
        \centering

        \raisebox{\dimexpr-\height+\ht\strutbox}{
          \includegraphics[height=0.08\textheight]{img/simulated/u0_mod}
        }
        \vspace{0.3em}
        \mycaption{
          (a) Ground truth image
        }
      \end{minipage}
      %
      \begin{minipage}[t]{0.49\textwidth}
        \centering

        \raisebox{\dimexpr-\height+\ht\strutbox}{
          \includegraphics[height=0.08\textheight]{img/simulated/f_gaussian_mod}
        }
        \vspace{0.3em}
        \mycaption{
          (b) Data 
        }
      \end{minipage}

      %\vspace{0.9em}
      \begin{minipage}[t]{0.49\textwidth}
        \centering

        \raisebox{\dimexpr-\height+\ht\strutbox}{
          \includegraphics[height=0.08\textheight]{img/simulated/gauss_urec_h_mod}
        }
        \vspace{-0.3em}
        \mycaption{
          (c) Constant PSF deconvolution 
        }
      \end{minipage}
      %
      \begin{minipage}[t]{0.49\textwidth}
        \centering

        \raisebox{\dimexpr-\height+\ht\strutbox}{
          \includegraphics[height=0.08\textheight]{img/simulated/gauss_urec_l_mod}
        }

        \vspace{0.3em}
        \mycaption{
          (d) Model deconvolution
        }
      \end{minipage}
    \end{minipage}
    %
    \hspace{-0.5em}
    \begin{minipage}[t]{0.45\textwidth}
      \begin{itemize}
        \item We use the total variation regulariser $TV(s)$
          and $\ell_2$ fidelity term due to the Gaussian
          noise in the data. 

        \item To solve the optimisation problem, we apply a version
          of the Primal Dual Hybrid Gradient (PDHG) 
          algorithm from \cite{Boulanger2018,Condat2013}.

        \item In the figure on the left, we show an example of a simulated sample (a) 
          to which we apply the forward model with Gaussian noise (b)
          and the result of deconvolving using only a spatially varying
          PSF (c) and the full forward model (d). 

        \item The images are shown 
          using maximum intensity projection.
      \end{itemize}
    \end{minipage}
}

\headerbox{Results}
{name=resultsreal,column=0, below=model, span=6}{
  \begin{minipage}[t]{0.51\textwidth} 
    \begin{center}
      \larger
      \textbf{\textsc{Beads}}
    \end{center}

   \hspace{-2em}
    \begin{minipage}[t]{0.49\textwidth}
      \begin{itemize}
        \item We apply the proposed method to a sample of 
          beads in agarose.
        \item The dimensions of the sample are 1127 x 111 x 100 pixels.

        \item
      \end{itemize}
    \end{minipage}
    %
    \begin{minipage}[t]{0.49\textwidth}
      \centering
      \raisebox{\dimexpr-\height+\ht\strutbox}{
        \includegraphics[width=\textwidth]{img/fBeads_xy1.png}
      }
      \raisebox{\dimexpr-\height+\ht\strutbox}{
        \includegraphics[width=\textwidth]{img/urecBeads_h_xy1.png}
      }
      \raisebox{\dimexpr-\height+\ht\strutbox}{
        \includegraphics[width=\textwidth]{img/urecBeads_xy1.png}
      }

      \vspace{-1em}
      \begin{center}
        \mycaption{
          XY slice: 
          data (top), constant PSF deconvolution (middle),
          model deconvolution (bottom)
        }
      \end{center}

      \vspace{-0.5em}
      \centering
      \raisebox{\dimexpr-\height+\ht\strutbox}{
        \includegraphics[width=\textwidth]{img/fBeads_xz1.png}
      }
      \raisebox{\dimexpr-\height+\ht\strutbox}{
        \includegraphics[width=\textwidth]{img/urecBeads_h_xz1.png}
      }
      \raisebox{\dimexpr-\height+\ht\strutbox}{
        \includegraphics[width=\textwidth]{img/urecBeads_xz1.png}
      }

      \vspace{-1em}
      \begin{center}
        \mycaption{
          XY slice:
          data (top), constant PSF deconvolution (middle),
          model deconvolution (bottom)
        }
      \end{center}
    \end{minipage}
  \end{minipage}
  %
  \begin{minipage}[t]{0.48\textwidth}
    \begin{center}
      \larger
      \textbf{\textsc{Marchantia}}
    \end{center}

    \hspace{-2em}
    \begin{minipage}[t]{0.5\textwidth}

      \begin{itemize}
        \item We apply the proposed method to a sample of 
          Marchantia plant.
        \item The dimensions of the sample 
          are 1127 x 155 x 100 pixels.

        \item
      \end{itemize}
    \end{minipage}
    %
    \begin{minipage}[t]{0.49\textwidth}
      %\hspace{0.6cm}
      \centering
      \raisebox{\dimexpr-\height+\ht\strutbox}{
        \includegraphics[width=\textwidth]{img/f_xy1.png}
      }
      \raisebox{\dimexpr-\height+\ht\strutbox}{
        \includegraphics[width=\textwidth]{img/urec_h_xy1.png}
      }
      \raisebox{\dimexpr-\height+\ht\strutbox}{
        \includegraphics[width=\textwidth]{img/m_urec_it700_lam0p1_xy1.png}
      }

      \vspace{-1em}
      \begin{center}
        \mycaption{
          XY slice: data (top), constant PSF deconvolution (middle),
          model deconvolution (bottom)
        }
      \end{center}

      \vspace{-0.8em}
      \centering
      \raisebox{\dimexpr-\height+\ht\strutbox}{
        \includegraphics[width=\textwidth]{img/f_xz1.png}
      }
      \raisebox{\dimexpr-\height+\ht\strutbox}{
        \includegraphics[width=\textwidth]{img/urec_h_xz1.png}
      }
      \raisebox{\dimexpr-\height+\ht\strutbox}{
        \includegraphics[width=\textwidth]{img/m_urec_it700_lam0p1_xz.png}
      }

      \vspace{-1em}
      \begin{center}
        \mycaption{
          XZ slice: data (top), constant PSF deconvolution (middle),
          model deconvolution (bottom)
        }
      \end{center}
    \end{minipage}
  \end{minipage}
  \vspace{-1em}
}

\headerbox{References}
{name=references, column=0,  below=resultsreal, span=4}
%{name=references, column=0,  above=bottom, span=1}
{
  %\smaller                                  % Make the whole text smaller
  %\footnotesize
  %\scriptsize
  \tiny
  \bibliographystyle{abbrv}                 % Use plain style
  \renewcommand{\section}[2]{\vspace{0.01em}}	% Omit "References" title
  \bibliography{references}
}

\headerbox{Acknowledgments}
{name=acknowledgments, column=4, below=resultsreal, span=2}
%{name=acknowledgments, column=1, above=bottom, span=1}
{  
  \scriptsize
  %\tiny
  This work is funded by 
  Isaac Newton Trust/Wellcome Trust ISSF/University of Cambridge 
  Joint Research Grants Scheme, RG89305.

  \par
}






\end{poster}
\end{document}

%%% Local Variables:
%%% mode: latex
%%% TeX-master: t
%%% End:
